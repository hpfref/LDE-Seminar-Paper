\documentclass[sigconf]{acmart}

\AtBeginDocument{%
  \providecommand\BibTeX{{%
    Bib\TeX}}}

%dummy copyright and reference data
\setcopyright{acmlicensed}
\copyrightyear{2018}
\acmYear{2018}
\acmDOI{XXXXXXX.XXXXXXX}
\acmConference[Conference acronym 'XX]{Make sure to enter the correct
  conference title from your rights confirmation emai}{June 03--05,
  2018}{Woodstock, NY}
\acmISBN{978-1-4503-XXXX-X/18/06}

\begin{document}


\title{Summary Paper of \\ \textit{Don’t Hold My Data Hostage – \\A Case For Client Protocol Redesign}}
\subtitle{Authors of Original Paper: Mark Raasveldt and Hannes Mühleisen}


\author{Hannes Pohnke}
%\authornote{Authors of summarized Paper: Mark Raasveldt and Hannes Mühleisen.}
\email{pohnke@tu-berlin.de}
\affiliation{
  \institution{TU Berlin}
  \city{Berlin}
  \country{Germany}
}


\begin{abstract}
Traditionally, database query processing and ML tasks are executed on separate, dedicated systems, but the current trend goes towards integrated data analysis pipelines that combine both tasks. In state of the art systems, orchestration of those two tasks still is inefficient due to expensive data transfer and missed global optimization potential. The paper we are summarizing, "Don’t Hold My Data Hostage – A Case For Client Protocol Redesign" by Mark Raasveldt and Hannes Muhleisen, addresses this problem by investigating the high cost of transferring large data from databases to the client programs, which can be much more time consuming than the actual query execution. The authors explore and analyse current serialization methods, that are used in database systems and identify their inefficiencies through various experiments. They also introduce a new columnar serialization method that can significantly enhance data transfer performance. By improving the data transfer, this approach could be a step towards efficiently combining database and machine learning systems.
\end{abstract}

\keywords{Databases, Client Protocols, Data Export}

\maketitle

\section{Introduction}
% Your introduction content here

\section{Summary}
% Your summary content here

\section{Analysis}
% Your analysis content here

% References
\bibliographystyle{ACM-Reference-Format}
\bibliography{references.bib}

\end{document}
